\documentclass{standalone}
\usepackage{auto-pst-pdf}

\pagestyle{empty}
\usepackage{pst-labo}
\usepackage{verbatim}
\begin{comment}
:Title: Das schwärzen von Kupfer
:Author: Robin Schneider https://ypid.wordpress.com/
\end{comment}

\begin{document}

\newcommand{\SKohlt}{black!50!brown}
\newpsstyle{VerkohltMehl}{linestyle=none,fillstyle=solid,fillcolor=\SKohlt}
\definecolor{Cu}{rgb}{.8,.7,.2}
\def\pstKupferDiff{{%
\pscurve[unit=0.5,fillstyle=gradient,gradmidpoint=0,linestyle=none,gradbegin=Cu,gradend=black]
(0,0)(-0.25,0.25)(0,.5)(0.25,0.75)(0.5,0.5)(0.15,0.4)(0.,0)}}

\newpsstyle{grau}{linestyle=none,fillstyle=solid,fillcolor=gray!60}
\begin{pspicture}(1,2.5)
\psset{unit=0.58cm}
\rput(-0.5,-5.59){\PSTBun}
\rput(-0.5,-0.15){\psellipse[fillcolor=\SKohlt,fillstyle=solid,linecolor=\SKohlt](0,-0.2)(0.3,0.15)}
\psset{unit=0.9cm}
\rput{135}(0.8,-0.12){\pstKupferDiff}	%% Cu
\psset{unit=0.58cm}
\rput(-0.1,2){\pstTubeEssais[%solide={\pstBullesChampagne[20]}, %% \pstBULLES[20]{white}
bouchon=true,tubePenche=-90,niveauLiquide1=0]}
%\psline[unit=0.5,linecolor=Cu](-1,-0.8)(0.5,-0.7)
\end{pspicture}

\end{document}
