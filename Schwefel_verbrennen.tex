\documentclass{standalone}
\usepackage{auto-pst-pdf}

\pagestyle{empty}
\usepackage{pst-labo}
\usepackage{verbatim}
\begin{comment}
:Title: Schwefel verbrennen; pH-Wert von Rauch wird in Lackmuslösung gezeigt
:Author: Robin Schneider https://ypid.wordpress.com/
Es entsteht Schwefeltrioxid (SO3), aufgrund des rostigem Verbrennungslöffels
\end{comment}

\begin{document}

\newpsstyle{redblue}{linestyle=none,fillstyle=solid,fillcolor=red!70}

\begin{pspicture}(0,11)
\psset{unit=0.58cm}
\pstTubeEssais[bouchon=true,glassType=erlen,niveauLiquide1=25,aspectLiquide1=redblue]
\psline[linewidth=0.1]{-}(-1.5,6.5)(-1.5,4)
\psline[linewidth=0.1]{-}(-1.5,3.1)(-1.5,1.3)
\psset{unit=0.3cm,xunit=1.4}
\rput(-2.2,2.1){\PSTflammeBlue}
\psset{unit=0.58cm,xunit=1}
\rput(-1.5,1.3){\pswedge[fillcolor=lightgray,linewidth=0.1,fillstyle=solid]{0.4}{180}{0}}
\end{pspicture}

\end{document}
